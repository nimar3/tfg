\documentclass[11pt,spanish,listoffigures,listoftables]{tfgetsinf}

\usepackage{enumitem}
\newlist{tabitem}{itemize}{1}
\setlist[tabitem]{wide=0pt, nosep, leftmargin= * ,label=\textbullet,after=\vspace{-\baselineskip},before=\vspace{-0.6\baselineskip}}

\usepackage[utf8]{inputenc} 


\title{Implementación de un sitio web para un concurso de programación paralela}
\author{Nicolás Fernando Martini}
\tutor{Pedro Alonso Jordá}
\curs{2019-2020}


\keywords{????, ?????????, ????, ?????????????????} % Paraules clau 
         {programación paralela, concurso de programación, gamificación}              % Palabras clave
         {parallel programming, programming competition, gamification}        % Key words

\begin{document}

\begin{abstract}
?
\end{abstract}
\begin{abstract}[spanish]
?
\end{abstract}
\begin{abstract}[english]
?
\end{abstract}

\mainmatter

\chapter{Introducción}

????? ????????????? ????????????? ????????????? ????????????? ?????????????

\section{Motivación}

????? ????????????? ????????????? ????????????? ????????????? ????????????? 

\section{Objectivo}

????? ????????????? ????????????? ????????????? ????????????? ?????????????

\section{Estructura de la memoria}

\chapter{Estado del arte}

\section{Alternativas}

\section{Crítica a alternativas}

\chapter{Análisis del problema}

\section{Casos de uso}

\subsection{Actores}

El sistema contará con dos perfiles de usuarios que llamaremos actores, estos son el Estudiante y el Administrador. También, veremos que en la memoria se hace mención al \textit{usuario}, esto se hace para referirnos a funcionalidades que se aplican tanto a los \textit{Estudiantes} como \textit{Administradores}

\begin{itemize}
	\item Estudiante: 
	\item Administrador:
\end{itemize}


\section{Historias de usuario}

\begin{itemize}
  \item Como \textit{Usuario}, quiero poder entrar al sistema utilizando mis credenciales personales.
  \item Como \textit{Estudiante}, quiero poder registrarme en el sistema para poder hacer uso del mismo.
  \item Como \textit{Estudiante}, quiero poder enviar las resoluciones a los tareas planteadas.
  \item Como \textit{Estudiante}, quiero poder formar parte de un equipo.
  \item Como \textit{Estudiante}, quiero poder visualizar un resumen de mi perfil. 
  \item Como \textit{Estudiante}, 
\end{itemize}

\section{Requisitos funcionales}

En el siguiente apartado se describen todos los requisitos funcionales de Pizarra.

\begin{table}[h!]
	\centering
	\begin{tabular}{ |p{4cm}||p{10cm}|  }
		\multicolumn{2}{l}{\textbf{RF-1}} \\
		\hline
		Nombre   & Registro de usuarios\\
		\hline
		Descripción  & El sistema permitirá el registro de usuarios mediante dirección de correo electrónico   \\
		\hline
		Prioridad &  Media\\
		\hline
		Criterio de aceptación & 
		\begin{tabitem}
			\item no puede existir más de un usuario con el mismo correo electrónico
			\item se generará un usuario único con el alias del correo electrónico
			\item se podrá deshabilitar el registro de usuarios mediante configuración del sistema
			\item se podrá limitar el registro de usuarios a correos electrónicos que pertenezcan a dominios específicos (eg: @upv.es, @inf.upv.es)
		\end{tabitem} \\
		\hline
	\end{tabular}
	\caption{Table to test captions and labels}
	\label{table:1}
\end{table}

\begin{table}[h!]
	\centering
	\begin{tabular}{ |p{4cm}||p{10cm}|  }
		\multicolumn{2}{l}{\textbf{RF-2}} \\
		\hline
		Nombre   & Login de usuarios\\
		\hline
		Descripción  & El sistema permitirá el login de usuarios mediante dirección de correo electrónico o usuario  \\
		\hline
		Prioridad &  Más Alta\\
		\hline
		Criterio de aceptación & 
		\begin{tabitem}
			\item solo se podrá ingresar al sistema si la cuenta está activa
		\end{tabitem} \\
		\hline
	\end{tabular}
	\caption{Table to test captions and labels}
	\label{table:2}
\end{table}


\chapter{??? ???? ??????}

????? ????????????? ????????????? ????????????? ????????????? ????????????? 

\section{?? ???? ???? ? ?? ??}

????? ????????????? ????????????? ????????????? ????????????? ?????????????

%%%%%%%%%%%%%%%%%%%%%%%%%%%%%%%%%%%%%%%%%%%%%%%%%%%%%%%%%%%%%%%%%%%%%%%%%%%%%%%
%                                 CONCLUSIONS                                 %
%%%%%%%%%%%%%%%%%%%%%%%%%%%%%%%%%%%%%%%%%%%%%%%%%%%%%%%%%%%%%%%%%%%%%%%%%%%%%%%

\chapter{Conclusions}

????? ????????????? ????????????? ????????????? ????????????? ????????????? 

%%%%%%%%%%%%%%%%%%%%%%%%%%%%%%%%%%%%%%%%%%%%%%%%%%%%%%%%%%%%%%%%%%%%%%%%%%%%%%%
%                                BIBLIOGRAFIA                                 %
%%%%%%%%%%%%%%%%%%%%%%%%%%%%%%%%%%%%%%%%%%%%%%%%%%%%%%%%%%%%%%%%%%%%%%%%%%%%%%%

\begin{thebibliography}{10}

%%%%%%%%%%%%%%%%%%%%%%%%%%%%%%%%%%%%%%%%%%%%%%%%%%%%%%%%%%%%%%%%%%%%%%%%%%%%%%%
% MODEL D'ARTICLE                                                             %
%%%%%%%%%%%%%%%%%%%%%%%%%%%%%%%%%%%%%%%%%%%%%%%%%%%%%%%%%%%%%%%%%%%%%%%%%%%%%%%
\bibitem{light}
   Jennifer~S. Light.
   \newblock When computers were women.
   \newblock \textit{Technology and Culture}, 40:3:455--483, juliol, 1999.

%%%%%%%%%%%%%%%%%%%%%%%%%%%%%%%%%%%%%%%%%%%%%%%%%%%%%%%%%%%%%%%%%%%%%%%%%%%%%%%
% MODEL DE LLIBRE                                                             %
%%%%%%%%%%%%%%%%%%%%%%%%%%%%%%%%%%%%%%%%%%%%%%%%%%%%%%%%%%%%%%%%%%%%%%%%%%%%%%%
\bibitem{ifrah}
   Georges Ifrah.
   \newblock \textit{Historia universal de las cifras}.
   \newblock Espasa Calpe, S.A., Madrid, sisena edició, 2008.

%%%%%%%%%%%%%%%%%%%%%%%%%%%%%%%%%%%%%%%%%%%%%%%%%%%%%%%%%%%%%%%%%%%%%%%%%%%%%%%
% MODEL D'URL                                                                 %
%%%%%%%%%%%%%%%%%%%%%%%%%%%%%%%%%%%%%%%%%%%%%%%%%%%%%%%%%%%%%%%%%%%%%%%%%%%%%%%
\bibitem{WAR}
   Comunicat de premsa del Departament de la Guerra, 
   emés el 16 de febrer de 1946. 
   \newblock Consultat a 
   \url{http://americanhistory.si.edu/comphist/pr1.pdf}.

\end{thebibliography}
\cleardoublepage

%%%%%%%%%%%%%%%%%%%%%%%%%%%%%%%%%%%%%%%%%%%%%%%%%%%%%%%%%%%%%%%%%%%%%%%%%%%%%%%
%                           APÈNDIXS  (Si n'hi ha!)                           %
%%%%%%%%%%%%%%%%%%%%%%%%%%%%%%%%%%%%%%%%%%%%%%%%%%%%%%%%%%%%%%%%%%%%%%%%%%%%%%%

\APPENDIX

%%%%%%%%%%%%%%%%%%%%%%%%%%%%%%%%%%%%%%%%%%%%%%%%%%%%%%%%%%%%%%%%%%%%%%%%%%%%%%%
%                         LA CONFIGURACIO DEL SISTEMA                         %
%%%%%%%%%%%%%%%%%%%%%%%%%%%%%%%%%%%%%%%%%%%%%%%%%%%%%%%%%%%%%%%%%%%%%%%%%%%%%%%

\chapter{Configuració del sistema}

????? ????????????? ????????????? ????????????? ????????????? ?????????????

\section{Fase d'inicialització}

????? ????????????? ????????????? ????????????? ????????????? ?????????????

\section{Identificació de dispositius}

????? ????????????? ????????????? ????????????? ????????????? ?????????????

%%%%%%%%%%%%%%%%%%%%%%%%%%%%%%%%%%%%%%%%%%%%%%%%%%%%%%%%%%%%%%%%%%%%%%%%%%%%%%%
%                               ALTRES  APÈNDIXS                              %
%%%%%%%%%%%%%%%%%%%%%%%%%%%%%%%%%%%%%%%%%%%%%%%%%%%%%%%%%%%%%%%%%%%%%%%%%%%%%%%


\chapter{??? ???????????? ????}

????? ????????????? ????????????? ????????????? ????????????? ????????????? 



%%%%%%%%%%%%%%%%%%%%%%%%%%%%%%%%%%%%%%%%%%%%%%%%%%%%%%%%%%%%%%%%%%%%%%%%%%%%%%%
%                              FI DEL DOCUMENT                                %
%%%%%%%%%%%%%%%%%%%%%%%%%%%%%%%%%%%%%%%%%%%%%%%%%%%%%%%%%%%%%%%%%%%%%%%%%%%%%%%

\end{document}
