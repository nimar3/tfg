\documentclass[11pt,spanish,listoffigures,listoftables]{tfgetsinf}

\usepackage[utf8]{inputenc} 

% custom bulletpoint for tables
\usepackage{enumitem}
\newlist{tabitem}{itemize}{1}
\setlist[tabitem]{wide=0pt, nosep, leftmargin= * ,label=\textbullet,after=\vspace{-\baselineskip},before=\vspace{-0.6\baselineskip}}
%

% custom dedication
\newenvironment{dedication}
{%\clearpage           % we want a new page          %% I commented this
	\thispagestyle{empty}% no header and footer
	\itshape             % the text is in italics
}
{\par % end the paragraph
	\vspace{\stretch{3}} % space at bottom is three times that at the top
	\clearpage           % finish off the page
}

% link colors in table black
\hypersetup{%
	colorlinks = true,
	linkcolor  = black
}

\newcommand\YAMLcolonstyle{\color{red}\mdseries}
\newcommand\YAMLkeystyle{\color{black}\bfseries}
\newcommand\YAMLvaluestyle{\color{blue}\mdseries}

\makeatletter

% here is a macro expanding to the name of the language
% (handy if you decide to change it further down the road)
\newcommand\language@yaml{yaml}

\expandafter\expandafter\expandafter\lstdefinelanguage
\expandafter{\language@yaml}
{
	keywords={true,false,null,y,n},
	keywordstyle=\color{darkgray}\bfseries,
	basicstyle=\YAMLkeystyle,                                 % assuming a key comes first
	sensitive=false,
	comment=[l]{\#},
	morecomment=[s]{/*}{*/},
	commentstyle=\color{purple}\ttfamily,
	stringstyle=\YAMLvaluestyle\ttfamily,
	moredelim=[l][\color{orange}]{\&},
	moredelim=[l][\color{magenta}]{*},
	moredelim=**[il][\YAMLcolonstyle{:}\YAMLvaluestyle]{:},   % switch to value style at :
	morestring=[b]',
	morestring=[b]",
	literate =    {---}{{\ProcessThreeDashes}}3
	{>}{{\textcolor{red}\textgreater}}1     
	{|}{{\textcolor{red}\textbar}}1 
	{\ -\ }{{\mdseries\ -\ }}3,
}

% switch to key style at EOL
\lst@AddToHook{EveryLine}{\ifx\lst@language\language@yaml\YAMLkeystyle\fi}
\makeatother

\newcommand\ProcessThreeDashes{\llap{\color{cyan}\mdseries-{-}-}}

\title{Implementación de un sitio web para un concurso de programación paralela}
\author{Nicolás Fernando Martini}
\tutor{Pedro Alonso Jordá}
\curs{2019-2020}


\keywords{?, ?, ?, ?} 
         {programación paralela, concurso de programación, gamificación, ludificación}  
         {parallel programming, programming competition, gamification}        

\begin{document}
	

\begin{dedication}	
	
	\chapter*{Dedicatoria}
	
	...

	\chapter*{Agradecimientos}
		A mi tutor Pedro Alonso Jordá por su apoyo y la oportunidad de hacer este proyecto con la esperanza que produzca un impacto positivo en la enseñanza en los cursos venideros. \par
		A la ETSINF y todos sus profesores, gracias a ellos he podido crecer personal y profesionalmente. \par
		Al Ministerio de Educación y Gobierno de España, gracias a sus ayudas he podido acceder los estudios de grado.
		
\end{dedication}

\begin{abstract}

... \par

... \par

... \par

\end{abstract}

\begin{abstract}[spanish]
	
El presente trabajo aborda la implementación de un sitio web para concursos de programación paralela, con el añadido de la gamificación o ludificación, una técnica de aprendizaje que busca recompensar al usuario y aumentar su motivación al sumar elementos y dinámicas propias de los juegos para así ofrecer una experiencia enriquecedora y positiva. \par 

Esta es una tarea compleja, por un lado llevar el proceso de envío de código a un entorno web y la interacción que tendrá con el cluster Kahan del \textit{DSIC}. Y por el otro, transformar las actividades de laboratorio de las asignaturas Coputación Paralela y Lenguajes y Entornos de la Programación Paralela con nuevas mecánicas que produzcan al estudiante buscar mejorar sus resultados inclusive después de llegar a una resolución correcta a los ejercicios planteados. \par

Este proyecto ha sido realizado con el apoyo del \textit{DSIC} de la \textit{UPV} con la finalidad de complementar otras herramientas utilizadas hoy en día para la enseñanza. \par


\end{abstract}

\begin{abstract}[english]

... \par

... \par

... \par

\end{abstract}

\mainmatter

\chapter{Introducción}

En este proyecto se trabajarán dos conceptos importantes, la programación paralela y la gamificación para llevarlos a un terreno en conjunto y así crear una herramienta que motive a los estudiantes en el aprendizaje de la asignatura de \textit{LPP}.

La programación paralela es una rama importante de la computación donde se buscar partir problemas de gran magnitud en pedazos más pequeños donde cada partición es ejecutada de forma simultanea por ...

La gamificación como herramienta para motivar ....

\section{Motivación}

Desde el inicio de mis estudios en el Grado de Ingeniería Informática me han interesado los concursos de programación como los promovidos desde la propia \textit{ETSINF} así como también los disponibles en plataformas online como UVa\footnote{UVa Online Judge website: \url{https://onlinejudge.org/}}, HackerRank\footnote{HackerRank website: \url{https://www.hackerrank.com/}}, CheckIO\footnote{CheckIO website: \url{https://checkio.org/}} y Project Euler\footnote{Project Euler website: \url{https://projecteuler.net/}}. Estas competiciones ayudan a promover el interés en diferentes ramas de los estudios cursados con el añadido de un marco competitivo. \par

Normalmente, los concursos de programación abarcan problemas relacionados con algoritmia, y al ver la publicación de la propuesta del \textit{TFG} donde se añade la programación paralela me ha despertado el interés ya que no solo se busca que se encuentre la solución al problema planteado, sino que también el punto más importante es la eficiencia. \par

Todo esto, además de poder contribuir en crear una herramienta que utilizarán los alumnos de la universidad me ha motivado a realizar esta aplicación que expongo en la memoria. 

\section{Objetivo}

El principal objetivo es crear una herramienta ....

\section{Estructura de la memoria}

La memoria está dividida en los siguientes capítulos:

\begin{itemize}
	\item \textbf{Estado del arte}: se analizan las alternativas que existen al trabajo planteado.  
	\item \textbf{Análisis del problema}: se recopila los requerimientos del proyecto para así llegar a una propuesta que abarque las necesidades actuales de la asignatura \textit{LPP}.
	\item \textbf{Diseño de la solución}: se describe todos los elementos a nivel de arquitectura que forman parte de la solución.
	\item \textbf{Desarrollo de la solución}: ?
	\item \textbf{Implantación}: se explica los pasos a seguir para desplegar el software.
	\item \textbf{Pruebas}: se describe las pruebas que se han realizado para verificar el correcto funcionamiento.
	\item \textbf{Mantenimiento}: ? 
	\item \textbf{Conclusiones}: ?
\end{itemize}

\chapter{Estado del arte}

\section{Crítica al estado del arte}

\section{Propuesta}


\chapter{Análisis del problema}

Antes de comenzar a escribir una linea de código, como todo proyecto ingenieril se debe hacer un estudio que involucre los usuarios que utilizarán el sistema donde se recogerán requisitos para poder hacer una propuesta acorde a sus necesidades.

\section{Actores}

El sistema contará con dos perfiles de usuarios que llamaremos actores, estos son el \textit{Estudiante} y \textit{Administrador}. También, veremos que en la memoria se hace mención al \textit{usuario}, esto se hace para referirnos a funcionalidades que se aplican tanto a los \textit{Estudiantes} como \textit{Administradores}

\begin{itemize}
	\item Estudiante: el rol principal del sistema. Enviará soluciones a los problemas propuestos y tratará de mejorar sus resultados para subir su ranking en la tabla de posiciones de las tareas y la grupal. Su trabajo será recompensado en forma de puntos e insignias.
	\item Administrador: el rol que se encarga de poner el sistema en marcha. Tendrá que tener conocimientos sólidos en la asignatura de \textit{CPA} para poder crear tareas que varíen en dificultad y motive al alumnado a mejorar sus resultados.
\end{itemize}


\section{Historias de usuario}

Como primer paso se describe el resultado esperado de este proyecto en un lenguaje sencillo que más adelante darán partida a los requisitos funcionales y no funcionales de la aplicación. A esto se lo llaman \textbf{Historias de usuario}.

\begin{itemize}
  \item Como \textit{Estudiante}, quiero ingresar al sistema utilizando mis credenciales personales.
  \item Como \textit{Estudiante}, quiero poder cambiar mi contraseña.
  \item Como \textit{Estudiante}, quiero registrarme en el sistema para poder hacer uso del mismo.
  \item Como \textit{Estudiante}, quiero enviar las resoluciones a mis tareas asignadas.
  \item Como \textit{Estudiante}, quiero formar parte de un equipo con otros estudiantes.
  \item Como \textit{Estudiante}, quiero acceder a mi perfil para ver toda mi información.
  \item Como \textit{Estudiante}, quiero visualizar los resultados de las ejecuciones para ver como han quedado posicionadas en la tabla de clasificación.
  \item Como \textit{Estudiante}, quiero ver un resumen de mi actividad.
  \item Como \textit{Administrador}, quiero crear grupos para que los estudiantes puedan formar parte de ellos.
  \item Como \textit{Administrador}, quiero crear estudiantes subiendo un fichero csv o txt.
  \item Como \textit{Administrador}, quiero crear y asignar tareas a diferentes grupos.
  \item Como \textit{Administrador}, quiero crear insignias para asignar a diferentes tareas.
  \item Como \textit{Administrador}, quiero editar y eliminar grupos, estudiantes, tareas e insignias.
  \item Como \textit{Administrador}, quiero desplegar el aplicativo en una solución \foreignlanguage{english}{cloud}.
\end{itemize}

\section{Requisitos funcionales}

En el siguiente apartado se describen todos los requisitos funcionales del aplicativo. \par

... explicación de la tabla ... \par

\begin{table}[h]
	\centering
	\begin{tabular}{ |p{4cm}||p{10cm}|  }
		\multicolumn{2}{l}{\textbf{RF-1}} \\
		\hline
		Nombre   & Registro de usuarios\\
		\hline
		Descripción  & El sistema permitirá el registro de usuarios mediante dirección de correo electrónico   \\
		\hline
		Prioridad &  Media\\
		\hline
		Criterio de aceptación & 
		\begin{tabitem}
			\item no puede existir más de un usuario con el mismo correo electrónico
			\item se generará un usuario único con el alias del correo electrónico
			\item se podrá deshabilitar el registro de usuarios mediante configuración del sistema
			\item se podrá limitar el registro de usuarios a correos electrónicos que pertenezcan a dominios específicos (eg: @upv.es, @inf.upv.es)
		\end{tabitem} \\
		\hline
	\end{tabular}
	\caption{RF-1 Registro de usuarios}
	\label{table:1}
\end{table}

\begin{table}[h]
	\centering
	\begin{tabular}{ |p{4cm}||p{10cm}|  }
		\multicolumn{2}{l}{\textbf{RF-2}} \\
		\hline
		Nombre   & Login de usuarios\\
		\hline
		Descripción  & El sistema permitirá el login de usuarios mediante dirección de correo electrónico o usuario  \\
		\hline
		Prioridad &  Más Alta\\
		\hline
		Criterio de aceptación & 
		\begin{tabitem}
			\item solo se podrá ingresar al sistema si la cuenta está activa
		\end{tabitem} \\
		\hline
	\end{tabular}
	\caption{RF-2 Login de usuarios}
	\label{table:2}
\end{table}

\begin{table}
	\centering
	\begin{tabular}{ |p{4cm}||p{10cm}|  }
		\multicolumn{2}{l}{\textbf{RF-3}} \\
		\hline
		Nombre   & Equipos\\
		\hline
		Descripción  & Los \textit{Estudiantes} podrán formar parte de un equipo con otros compañeros de grupo  \\
		\hline
		Prioridad &  Baja\\
		\hline
		Criterio de aceptación & 
		\begin{tabitem}
			\item solo se podrá formar parte de un único equipo en un momento determinado de tiempo
			\item si un equipo no tiene integrantes se eliminará del sistema
			\item los equipos contarán con una URL única que permitirá a otros estudiantes unirse a los mismos
			\item un \textit{Estudiante} no puede unirse un equipo que no forme parte de su grupo o que haya llegado al máximo de integrantes
			\item los equipos tendrán un máximo de integrantes que podrá ser configurado por el \textit{Administrador}
		\end{tabitem} \\
		\hline
	\end{tabular}
	\caption{RF-3 Equipos}
	\label{table:3}
\end{table}


\begin{table}
	\centering
	\begin{tabular}{ |p{4cm}||p{10cm}|  }
		\multicolumn{2}{l}{\textbf{RF-4}} \\
		\hline
		Nombre   & \foreignlanguage{english}{Dashboard}\\
		\hline
		Descripción  & Los \textit{Estudiantes} al ingresar verán un resumen agregado de su estado general en el sistema   \\
		\hline
		Prioridad &  Alta\\
		\hline
		Criterio de aceptación & El \foreignlanguage{english}{Dashboard} o \textit{Página de Inicio} del \textit{Estudiante} deberá mostrar la siguiente información: \newline
		\begin{tabitem}
			\item resumen de los últimos envíos
			\item extracto calculado de cantidad de envíos, tareas, quota disponible y puntaje
			\item la última insignia obtenida, en caso que no tuviere, alentarlo a completar una tarea para conseguir su primera
			\item los integrantes del equipo, en caso que no tuviere, alentarlo a crear uno nuevo o unirse a uno existente
		\end{tabitem} \\
		\hline
	\end{tabular}
	\caption{RF-4 \foreignlanguage{english}{Dashboard}}
	\label{table:4}
\end{table}

\begin{table}
	\centering
	\begin{tabular}{ |p{4cm}||p{10cm}|  }
		\multicolumn{2}{l}{\textbf{RF-5}} \\
		\hline
		Nombre   & Mis Envíos \\
		\hline
		Descripción  & Los \textit{Usuarios} podrán ver un resumen de sus envíos enviados al sistema   \\
		\hline
		Prioridad &  Muy Alta\\
		\hline
		Criterio de aceptación & El resumen de envíos se visualizará en forma de tabla, se podrá filtrar por la información disponible y deberá mostrar las siguientes columnas: \newline
		\begin{tabitem}
			\item ID único en el sistema
			\item Fecha y Hora de envío
			\item Tarea a la que corresponde
			\item Estado del envío
			\item Puntaje obtenido
			\item Tiempo de ejecución
		\end{tabitem} \\
		\hline
	\end{tabular}
	\caption{RF-5 Mis Envíos}
	\label{table:5}
\end{table}

\begin{table}
	\centering
	\begin{tabular}{ |p{4cm}||p{10cm}|  }
		\multicolumn{2}{l}{\textbf{RF-6}} \\
		\hline
		Nombre   & Envío \\
		\hline
		Descripción  & Visualización del envío a ejecutar  \\
		\hline
		Prioridad &  Alta\\
		\hline
		Criterio de aceptación & Se deben mostrar los siguientes apartados e información: \newline  
		\begin{tabitem}
			\item código fuente enviado
			\item análisis estático del código fuente enviado
			\item output del código ejecutado
			\item tiempo de ejecución y status
		\end{tabitem} \\
		\hline
	\end{tabular}
	\caption{RF-6 Envío}
	\label{table:6}
\end{table}

\begin{table}
	\centering
	\begin{tabular}{ |p{4cm}||p{10cm}|  }
		\multicolumn{2}{l}{\textbf{RF-7}} \\
		\hline
		Nombre   & Tareas \\
		\hline
		Descripción  & Visualización de listado de Tareas \\
		\hline
		Prioridad &  Alta\\
		\hline
		Criterio de aceptación & Se deben mostrar los siguientes apartados e información: \newline  
		\begin{tabitem}
			\item si el usuario es un \textit{Estudiante} se deben mostrar las tareas asignadas a su grupo
			\item si el usuario es un \textit{Administrador} se deben mostrar todas las tareas en el sistema
			\item por cada Tarea debe haber un resumen, insignias y últimos envíos
			\item no se deben mostrar las insignias secretas
		\end{tabitem} \\
		\hline
	\end{tabular}
	\caption{RF-7 Visualización de Tareas}
	\label{table:7}
\end{table}

\begin{table}
	\centering
	\begin{tabular}{ |p{4cm}||p{10cm}|  }
		\multicolumn{2}{l}{\textbf{RF-8}} \\
		\hline
		Nombre   & Tarea \\
		\hline
		Descripción  & Visualización de Tarea \\
		\hline
		Prioridad &  Alta\\
		\hline
		Criterio de aceptación & Se deben mostrar los siguientes apartados e información: \newline  
		\begin{tabitem}
			\item descripción completa
			\item insignias a obtener
			\item insignias secretas que el usuario ha obtenido
			\item status de la tarea: abre pronto, abierto, cierra pronto, cerrada
			\item si el usuario ha enviado una ejecución satisfactoria se le debe informar en un mensaje
			\item enlace para hacer un envío con sus credenciales
			\item enlace a la tabla de posiciones
		\end{tabitem} \\
		\hline
	\end{tabular}
	\caption{RF-8 Visualización de Tarea}
	\label{table:8}
\end{table}

\begin{table}
	\centering
	\begin{tabular}{ |p{4cm}||p{10cm}|  }
		\multicolumn{2}{l}{\textbf{RF-9}} \\
		\hline
		Nombre   & FAQ \\
		\hline
		Descripción  & Página con preguntas frecuentes que se pueden hacer los usuarios del sistema y sus respuestas  \\
		\hline
		Prioridad &  Baja\\
		\hline
		Criterio de aceptación & 
		\begin{tabitem}
			\item debe estar localizado en al menos 2 idiomas
			\item no debe contener más de 10 preguntas
		\end{tabitem} \\
		\hline
	\end{tabular}
	\caption{RF-9 FAQ}
	\label{table:9}
\end{table}

\begin{table}
	\centering
	\begin{tabular}{ |p{4cm}||p{10cm}|  }
		\multicolumn{2}{l}{\textbf{RF-10}} \\
		\hline
		Nombre & Tabla de Posiciones \\
		\hline
		Descripción &  \\
		\hline
		Prioridad & Alta\\
		\hline
		Criterio de aceptación & 
		\begin{tabitem}
			\item ?
		\end{tabitem} \\
		\hline
	\end{tabular}
	\caption{RF-10 Tabla de Posiciones}
	\label{table:10}
\end{table}

\begin{table}
	\centering
	\begin{tabular}{ |p{4cm}||p{10cm}|  }
		\multicolumn{2}{l}{\textbf{RF-11}} \\
		\hline
		Nombre & Perfil \\
		\hline
		Descripción &  \\
		\hline
		Prioridad & Media\\
		\hline
		Criterio de aceptación & 
		\begin{tabitem}
			\item ?
		\end{tabitem} \\
		\hline
	\end{tabular}
	\caption{RF-11 Perfil}
	\label{table:11}
\end{table}


\section{Requisitos no funcionales}

Los requisitos no funcionales son criterios que se deben cumplir para juzgar la correcta operación del sistema. En contraste con los requisitos funcionales, no definen comportamientos específicos. \par

... explicación de la tabla ... \par

\begin{table}[h]
	\centering
	\begin{tabular}{ |p{4cm}||p{10cm}|  }
		\multicolumn{2}{l}{\textbf{RNF-1}} \\
		\hline
		Nombre   & Usabilidad \\
		\hline
		Descripción  & El sitio web deberá tener una interfaz sencilla y fácil de utilizar. \\
		\hline
		Prioridad &  Muy Alta \\
		\hline
	\end{tabular}
	\caption{RNF-1 Usabilidad}
	\label{table:24}
\end{table}

\begin{table}
	\centering
	\begin{tabular}{ |p{4cm}||p{10cm}|  }
		\multicolumn{2}{l}{\textbf{RNF-2}} \\
		\hline
		Nombre   & Implantación \\
		\hline
		Descripción  & El aplicativo debe poder implantarse de forma automatizada \\
		\hline
		Prioridad &  Media \\
		\hline
	\end{tabular}
	\caption{RNF-2 Implantación}
	\label{table:25}
\end{table}

\begin{table}
	\centering
	\begin{tabular}{ |p{4cm}||p{10cm}|  }
		\multicolumn{2}{l}{\textbf{RNF-3}} \\
		\hline
		Nombre   & Configuración \\
		\hline
		Descripción  & El aplicativo debe permitir configurar los parámetros de despliegue y uso \\
		\hline
		Prioridad &  Alta \\
		\hline
	\end{tabular}
	\caption{RNF-3 Configuración}
	\label{table:26}
\end{table}

\begin{table}
	\centering
	\begin{tabular}{ |p{4cm}||p{10cm}|  }
		\multicolumn{2}{l}{\textbf{RNF-4}} \\
		\hline
		Nombre   & Localización \\
		\hline
		Descripción  & El aplicativo debe soportar la localización en diferentes idiomas \\
		\hline
		Prioridad &  Media \\
		\hline
	\end{tabular}
	\caption{RNF-4 Localización}
	\label{table:27}
\end{table}



\chapter{Diseño de la solución}

?

\section{Software}

?

\section{Arquitectura del sistema}

Resumen de como es el sistema a nivel de arquitectura. Redis, PostgreSQL, etc..



\chapter{Desarrollo de la solución}

?

\section{Entregables}

Sprints ¿?

\chapter{Implantación}

Pizarra es una herramienta que necesita de otros aplicativos para su funcionamiento. Por un lado una \textit{BD} para la persistencia de datos, Redis para el sistema de colas interno de envío de tareas, Nginx para actuar como servidor web y redirigir las peticiones. En el pasado la implantación hubiera requerido la intervención del equipo de Administración de Sistemas que se encargue de instalar el \textit{SO} de cada máquina, crear usuarios, asignar permisos, abrir puertos y otro sinfín de tareas.

Para evitar todos estos pasos y acelerar la implantación el proyecto incluye la posibilidad de que todos los elementos de la arquitectura se desplieguen en contenedores. Estos contenedores están creados con la tecnología Docker y son paquetes que incluyen todo lo necesario, sean librerías y herramientas del sistema para que el software se ejecute.

\section{Docker y Kubernetes}

... explicación simple Docker ...

Por ejemplo, para nuestro entorno de desarrollo local, utilizamos contenedores públicos como el de Postgres que con un fichero de configuración nos permite tener servicios ejecutándose en cuestión de minutos.

A continuación, un extracto del fichero \textbf{docker-compose.yaml} del repositorio\footnote{Docker yaml: \url{https://github.com/nimar3/pizarra/blob/master/docker-compose.yml}} con una breve explicación de los parámetros.

\begin{lstlisting}[language=yaml]
version: "3.1"
services:
  postgres:
    image: postgres: 9.6-alpine
    container_name: postgres
    volumes:
      - /data/pizarra/postgres:/var/lib/postgresql/data
    ports:
      - 5432: 5432
    environment:
      - POSTGRES_USER=pizarra
      - POSTGRES_PASSWORD=pizarra
    restart: unless-stopped
\end{lstlisting}

\begin{itemize}
	\item \textbf{version}: versión de la API de Docker que utilizaremos.
	\item \textbf{services}: listado de servicios que desplegaremos.
	\item \textbf{image}: contenedor que utilizaremos, en este caso el oficial de postgres\footnote{Postgres Official Docker Image: \url{https://hub.docker.com/_/postgres}} sobre Linux Alpine.
	\item \textbf{container\_name}: nombre con el que comenzará el contenedor que se ejecute.
	\item \textbf{volumes}: mapeo de directorios de nuesto entorno de desarrollo local al contenedor, esto se hace para evitar perder la información de la \textit{BD} al ser los contenedores \foreignlanguage{english}{stateless}.
    \item \textbf{ports}: mapeo de puertos del \textit{SO} a los contenedores, esto permite acceder al servicio de Postgres por un puerto local.
	\item \textbf{environment}: variables de entorno del \textit{SO}, en nuestro caso definimos un usuario y contraseña para acceder a la \textit{BD}
	\item \textbf{restart}: con el parámetro \foreignlanguage{english}{unless-stopped} nuestro contenedor se reiniciará de forma automática en caso de errores a menos que se envíe un mensaje de finalización de ejecución.
\end{itemize}

Con los parámetros previos y siguiendo la documentación de variables de entorno disponibles, al iniciarse el contenedor se crearán los \foreignlanguage{english}{schemas} y directorios necesarios en caso de que no existan y un usuario con privilegios con el nombre y contraseña proporcionados.



\section{Despliegue en la Nube}

Existen varios proveedores de Kubernetes en Internet pudiendo consultar el listado de todos los partners en el sitio oficial\footnote{Kubernetes - Partners: \url{https://kubernetes.io/es/partners}}. Siendo los más importantes Google Cloud Engine, Amazon AWS y Microsoft Azure se ha escogido por desplegar la aplicación en \textit{GCE} ya que  dispone de buena documentación y además \$300 de crédito promocional que són más que suficientes para tener el aplicativo corriendo durante meses.

\subsection{Pasos de un despliegue en GCE}

Antes de poder comenzar con el despliegue, debemos crearnos una cuenta en Google o utilizar una existente que nunca ha activado \textit{GCE} de forma previa ya que el crédito promocional de bienvenida tiene una caducidad de 12 meses desde la activación.




\subsection{Costos asociados}



\chapter{Pruebas}

?

\chapter{Mantenimiento}

Resumen de como montar un entorno local de desarrollo, o hacer debug en el sistema.. etc 

\section{Licencia}

explicación licencia MIT 

\chapter{Conclusiones}

... \par

\section{Relación del trabajo desarrollado con los estudios cursados}

?

\chapter{Trabajos futuros}

?

\begin{thebibliography}{10}

%%%%%%%%%%%%%%%%%%%%%%%%%%%%%%%%%%%%%%%%%%%%%%%%%%%%%%%%%%%%%%%%%%%%%%%%%%%%%%%
% MODEL D'ARTICLE                                                             %
%%%%%%%%%%%%%%%%%%%%%%%%%%%%%%%%%%%%%%%%%%%%%%%%%%%%%%%%%%%%%%%%%%%%%%%%%%%%%%%
\bibitem{light}
   Jennifer~S. Light.
   \newblock When computers were women.
   \newblock \textit{Technology and Culture}, 40:3:455--483, juliol, 1999.

%%%%%%%%%%%%%%%%%%%%%%%%%%%%%%%%%%%%%%%%%%%%%%%%%%%%%%%%%%%%%%%%%%%%%%%%%%%%%%%
% MODEL DE LLIBRE                                                             %
%%%%%%%%%%%%%%%%%%%%%%%%%%%%%%%%%%%%%%%%%%%%%%%%%%%%%%%%%%%%%%%%%%%%%%%%%%%%%%%
\bibitem{ifrah}
   Georges Ifrah.
   \newblock \textit{Historia universal de las cifras}.
   \newblock Espasa Calpe, S.A., Madrid, sisena edició, 2008.

%%%%%%%%%%%%%%%%%%%%%%%%%%%%%%%%%%%%%%%%%%%%%%%%%%%%%%%%%%%%%%%%%%%%%%%%%%%%%%%
% MODEL D'URL                                                                 %
%%%%%%%%%%%%%%%%%%%%%%%%%%%%%%%%%%%%%%%%%%%%%%%%%%%%%%%%%%%%%%%%%%%%%%%%%%%%%%%
\bibitem{WAR}
   Comunicat de premsa del Departament de la Guerra, 
   emés el 16 de febrer de 1946. 
   \newblock Consultat a 
   \url{http://americanhistory.si.edu/comphist/pr1.pdf}.

\end{thebibliography}
\cleardoublepage

\APPENDIX

\chapter{Glosario}

\begin{itemize}
	\item \textbf{Tarea}:
	\item \textbf{Insignia}:
	\item \textbf{Equipo de Estudiantes}:
	\item \textbf{Grupo de Estudiantes}:
\end{itemize}

\chapter{Acrónimos}

\begin{itemize}
	\item \textbf{DSIC}: Departamento de Sistemas Informáticos y Computación.
	\item \textbf{UPV}: Universitat Politècnica de València.
	\item \textbf{ETSINF}: Escuela Técnica Superior de Ingeniería Informática.
	\item \textbf{CPA}: Computación Paralela.
	\item \textbf{LPP}: Lenguajes y Entornos de la Programación Paralela.
	\item \textbf{TFG}: Trabajo Final de Grado.
	\item \textbf{RF}: Requisito funcional.
	\item \textbf{RNF}: Requisito no funcional.
	\item \textbf{GCE}: Google Cloud Engine.
	\item \textbf{DB}: Base de datos.
	\item \textbf{SO} Sistema operativo.
\end{itemize}

\chapter{Configuración del sistema}

Parámetros disponibles para la configuración del sistema.

\section{Inicialización}

?

\chapter{?}

?

\end{document}
