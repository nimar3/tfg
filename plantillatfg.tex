\documentclass[11pt,spanish,listoffigures,listoftables]{tfgetsinf}

\usepackage[utf8]{inputenc} 

% custom bulletpoint for tables
\usepackage{enumitem}
\newlist{tabitem}{itemize}{1}
\setlist[tabitem]{wide=0pt, nosep, leftmargin= * ,label=\textbullet,after=\vspace{-\baselineskip},before=\vspace{-0.6\baselineskip}}
%

% custom dedication
\newenvironment{dedication}
{%\clearpage           % we want a new page          %% I commented this
	\thispagestyle{empty}% no header and footer
	\itshape             % the text is in italics
	\raggedleft          % flush to the right margin
}
{\par % end the paragraph
	\vspace{\stretch{3}} % space at bottom is three times that at the top
	\clearpage           % finish off the page
}

% link colors in table black
\hypersetup{%
	colorlinks = true,
	linkcolor  = black
}

\title{Implementación de un sitio web para un concurso de programación paralela}
\author{Nicolás Fernando Martini}
\tutor{Pedro Alonso Jordá}
\curs{2019-2020}


\keywords{????, ?????????, ????, ?????????????????} 
         {programación paralela, concurso de programación, gamificación, ludificación}  
         {parallel programming, programming competition, gamification}        

\begin{document}

\begin{dedication}

	... 
	
\end{dedication}

\begin{abstract}
?
\end{abstract}
\begin{abstract}[spanish]
?
\end{abstract}
\begin{abstract}[english]
?
\end{abstract}

\mainmatter

\chapter{Introducción}

?

\section{Motivación}

Desde el inicio de mis estudios en el Grado de Ingeniería Informática me han interesado los concursos de programación como los promovidos desde la propia \textit{ETSINF} así como también los disponibles en plataformas online como UVa, HackerRank, CheckIO y Project Euler. Estas competiciones ayudan a promover el interés en diferentes ramas de los estudios cursados con el añadido de un marco competitivo. 

Normalmente, los concursos de programación abarcan problemas relacionados con algoritmia, y al ver la publicación de la propuesta del \textit{TFG} donde se añade la programación paralela me ha despertado el interés ya que no solo se busca que se encuentre la solución al problema planteado, sino que también su eficiencia.

Además, es importante remarcar que hoy en día muchas empresas en sus entrevistas de trabajo recurren a este tipo de resolución de problemas para contratar a candidatos.

\section{Objetivo}

El principal objetivo es ...

\section{Estructura de la memoria}

La memoria está dividida en los siguientes capítulos:

\begin{itemize}
	\item \textbf{Estado del arte}: se analizan las alternativas que existen al trabajo planteado para así llegar a una propuesta que abarque las necesidades actuales de la asignatura Computación Paralela.
	\item \textbf{Análisis del problema}: se recopilan los tipos de usuarios que harán uso del sistema, los requerimientos que se deben implementar para que el sistema cumpla con los objetivos planteados.
	\item \textbf{Diseño de la solución}: se describen los elementos
	\item \textbf{Desarrollo de la solución}:
	\item \textbf{Implantación}:
	\item \textbf{Pruebas}:
	\item \textbf{Conclusiones}: 
\end{itemize}

\chapter{Estado del arte}

\section{Crítica al estado del arte}

\section{Propuesta}

\chapter{Análisis del problema}

\section{Casos de uso}

\subsection{Actores}

El sistema contará con dos perfiles de usuarios que llamaremos actores, estos son el \textit{Estudiante} y \textit{Administrador}. También, veremos que en la memoria se hace mención al \textit{usuario}, esto se hace para referirnos a funcionalidades que se aplican tanto a los \textit{Estudiantes} como \textit{Administradores}

\begin{itemize}
	\item Estudiante: 
	\item Administrador:
\end{itemize}


\section{Historias de usuario}

\begin{itemize}
  \item Como \textit{Usuario}, quiero ingresar al sistema utilizando mis credenciales personales.
  \item Como \textit{Estudiante}, quiero registrarme en el sistema para poder hacer uso del mismo.
  \item Como \textit{Estudiante}, quiero enviar las resoluciones a mis tareas asignadas.
  \item Como \textit{Estudiante}, quiero formar parte de un equipo.
  \item Como \textit{Estudiante}, quiero visualizar un resumen de mi perfil. 
  \item ...
\end{itemize}

\section{Requisitos funcionales}

En el siguiente apartado se describen todos los requisitos funcionales de Pizarra.

\begin{table}[h!]
	\centering
	\begin{tabular}{ |p{4cm}||p{10cm}|  }
		\multicolumn{2}{l}{\textbf{RF-1}} \\
		\hline
		Nombre   & Registro de usuarios\\
		\hline
		Descripción  & El sistema permitirá el registro de usuarios mediante dirección de correo electrónico   \\
		\hline
		Prioridad &  Media\\
		\hline
		Criterio de aceptación & 
		\begin{tabitem}
			\item no puede existir más de un usuario con el mismo correo electrónico
			\item se generará un usuario único con el alias del correo electrónico
			\item se podrá deshabilitar el registro de usuarios mediante configuración del sistema
			\item se podrá limitar el registro de usuarios a correos electrónicos que pertenezcan a dominios específicos (eg: @upv.es, @inf.upv.es)
		\end{tabitem} \\
		\hline
	\end{tabular}
	\caption{RF-1 Registro de usuarios}
	\label{table:1}
\end{table}

\begin{table}[h!]
	\centering
	\begin{tabular}{ |p{4cm}||p{10cm}|  }
		\multicolumn{2}{l}{\textbf{RF-2}} \\
		\hline
		Nombre   & Login de usuarios\\
		\hline
		Descripción  & El sistema permitirá el login de usuarios mediante dirección de correo electrónico o usuario  \\
		\hline
		Prioridad &  Más Alta\\
		\hline
		Criterio de aceptación & 
		\begin{tabitem}
			\item solo se podrá ingresar al sistema si la cuenta está activa
		\end{tabitem} \\
		\hline
	\end{tabular}
	\caption{RF-2 Login de usuarios}
	\label{table:2}
\end{table}

\begin{table}[h!]
	\centering
	\begin{tabular}{ |p{4cm}||p{10cm}|  }
		\multicolumn{2}{l}{\textbf{RF-2}} \\
		\hline
		Nombre   & Equipos\\
		\hline
		Descripción  & Los \textit{Estudiantes} podrán formar parte de un equipo   \\
		\hline
		Prioridad &  Baja\\
		\hline
		Criterio de aceptación & 
		\begin{tabitem}
			\item solo se podrá formar parte de un solo equipo
			\item si un equipo no tiene integrantes se eliminará del sistema
			\item los equipos tendrán un máximo de integrantes que podrá ser configurado por el \textit{Administrador}
		\end{tabitem} \\
		\hline
	\end{tabular}
	\caption{RF-3 Equipos}
	\label{table:3}
\end{table}


\chapter{Diseño de la solución}

?

\section{Arquitectura del sistema}

?

\chapter{Implantación}

?

\chapter{Pruebas}

?

\chapter{Conclusiones}

\begin{thebibliography}{10}

%%%%%%%%%%%%%%%%%%%%%%%%%%%%%%%%%%%%%%%%%%%%%%%%%%%%%%%%%%%%%%%%%%%%%%%%%%%%%%%
% MODEL D'ARTICLE                                                             %
%%%%%%%%%%%%%%%%%%%%%%%%%%%%%%%%%%%%%%%%%%%%%%%%%%%%%%%%%%%%%%%%%%%%%%%%%%%%%%%
\bibitem{light}
   Jennifer~S. Light.
   \newblock When computers were women.
   \newblock \textit{Technology and Culture}, 40:3:455--483, juliol, 1999.

%%%%%%%%%%%%%%%%%%%%%%%%%%%%%%%%%%%%%%%%%%%%%%%%%%%%%%%%%%%%%%%%%%%%%%%%%%%%%%%
% MODEL DE LLIBRE                                                             %
%%%%%%%%%%%%%%%%%%%%%%%%%%%%%%%%%%%%%%%%%%%%%%%%%%%%%%%%%%%%%%%%%%%%%%%%%%%%%%%
\bibitem{ifrah}
   Georges Ifrah.
   \newblock \textit{Historia universal de las cifras}.
   \newblock Espasa Calpe, S.A., Madrid, sisena edició, 2008.

%%%%%%%%%%%%%%%%%%%%%%%%%%%%%%%%%%%%%%%%%%%%%%%%%%%%%%%%%%%%%%%%%%%%%%%%%%%%%%%
% MODEL D'URL                                                                 %
%%%%%%%%%%%%%%%%%%%%%%%%%%%%%%%%%%%%%%%%%%%%%%%%%%%%%%%%%%%%%%%%%%%%%%%%%%%%%%%
\bibitem{WAR}
   Comunicat de premsa del Departament de la Guerra, 
   emés el 16 de febrer de 1946. 
   \newblock Consultat a 
   \url{http://americanhistory.si.edu/comphist/pr1.pdf}.

\end{thebibliography}
\cleardoublepage

\APPENDIX

\chapter{Acrónimos}

\begin{itemize}
	\item \textbf{ETSINF}: Escuela Técnica Superior de Ingeniería Informática.
	\item \textbf{TFG}: Trabajo Final de Grado.
	\item \textbf{RF}: Requisito funcional.
	\item \textbf{RNF}: Requisito no funcional.
\end{itemize}

\chapter{Configuración del sistema}

?

\section{Inicialización}

?

\chapter{?}

?

\end{document}
